\chapter{Introduction}
Queues with scheduled arrivals occur frequently in society. These are queues where instead of customers arriving randomly, their arrival times are scheduled in advance. A common example is a doctor's surgery where patients are given appointment times. These queues also occur in shipping when ship docking times are scheduled in advance.

In this thesis, we make several assumptions about the underlying system. First, we assume a single server queue with exponential service times. We assume that customers arrive punctually at their scheduled arrival times. In addition, all customers have the same mean service time $\mu$.

The objective is to find a schedule of customer arrival times that minimises a linear combination of the expected total waiting time of all the customers and the expected server availability time (i.e., the expected time between the start and end of service). This linear combination is referred to as the expected cost of a schedule.

A customer's waiting time is the time from the customer's arrival until the server begins serving that customer. Throughout this thesis, we refer to the number of customers in the system at a given point in time. This number includes both any customers currently being served and any customers who have arrived and are currently waiting for service.

Scheduling customers involves a trade off between customer waiting time and server availability time. Scheduling customers to arrive further apart decreases the expected waiting time of the customers as there are less customers in the system at the time of each customer's arrival, but increases the expected server availability time. On the other hand, scheduling customers to arrive closer together decreases the expected server availability time, but increases the expecter waiting time of the customers.

\citet{Bailey} was the first to study such queues. Bailey proposed a rule whereby the first two customers should be scheduled to arrive at the start of service and the remaining customers should be scheduled to arrive at fixed intervals. Bailey found that this rule reduced time wasted by customers without signficiantly increasing the server's idle time.

Many authors have extended on Bailey's ideas. \citet{Pegden} propose a method for determing the optimal scheduled arrival times. \citet{Mendel} extends the model to allow for no-shows, whereby a customer fails to arrive for service. \citet{Fiems} include emergency requests that immediately halt the server.

Unfortunately, few authors have considered the problem of adjusting a given schedule. Most authors assume that a schedule is fixed at the start of service and cannot be altered during service. This lack of freedom to alter a schedule could be a restrictive assumption. In this thesis, we examine the potential advantage of being able to adjust some scheduled arrival times during service.

In Chapter~\ref{chap:Static}, we derive a method for determing the optimal sequence of arrival times for a given number of customers. This method assumes that the schedule is fixed for the duration of service. We call such a schedule a `static schedule.' The results in this chapter largely follow the work of \citet{Pegden}.

In Chapter~\ref{chap:Dynamic}, we consider a special case where customer arrivals are scheduled iteratively (i.e., one by one). A customer's arrival time is only decided on the arrival of the customer to be served immediately before them. Such a schedule is called a `dynamic schedule.' This schedule is equivalent to the scheduler being able to freely adjust each customer's scheduled arrival time up until the arrival of the customer to be served immediately before them.

Towards the end of both Chapters~\ref{chap:Static} and~\ref{chap:Dynamic}, a number of example models are considered. These models help to understand the properties of each of the schedules. We examine the simple case of a small number of customers to be scheduled, and the more interesting case of a larger number of customers to be scheduled.

The dynamic schedule is simply a more general case of the static schedule. In Chapter~\ref{chap:Comparison}, we compare the expected cost of the two schedules and confirm this intuitive result. Moreover, we investigate the percentage difference between the expected cost of each schedule as the number of customers to be scheduled increases.

In Chapter~\ref{chap:Simulation}, we seek a broader understanding of the differences between the two schedules. We simulate a million runs of each schedule and compare the two cost distributions. The mean customer arrival time and customer waiting time under each schedule is examined. This chapter gives us a fuller understanding of the properties of each schedule that are difficult to derive analytically.