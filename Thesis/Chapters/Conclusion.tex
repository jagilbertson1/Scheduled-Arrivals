\chapter{Conclusion}
The goal of this thesis was to derive methods for determining the optimal customer arrival times in a single-server queue with punctual customers. In Chapter~\ref{chap:Static}, we derived a similar method to \citet{Pegden} for determining the optimal static schedule. In Chapter~\ref{chap:Dynamic}, we derived an iterative procedure for determining the optimal dynamic schedule. This procedure determines the optimal arrival time of the next customer given the number of customers still to be scheduled and the number of customers currently in the system. In both schedules, the optimal schedule minimises a linear combination of the expected total customer waiting time and the expected server availability time.

In Chapter~\ref{chap:Static}, we were able to derive algebraic expressions for the optimal policy for schedules with less than three customers. As expected, the interarrival time between the first two customers decreases as the server availability cost increases. In addition, we found numerical solutions for the case of scheduling $15$ customers. We observed the dome-shape, which is a commonly observed property of static schedules. The optimal interarrival times increase for the initial few customers, remain constant for the majority of customers, and decrease for the final few customers.

In Chapter~\ref{chap:Dynamic}, we derived exact algebraic expressions for the optimal dynamic schedule where there are less than three customers. These dynamic schedules exact match the optimal static schedules found in Chapter~\ref{chap:Static}. For more than two customers, the optimal dynamic schedule consists of the optimal policies at each of the possible states during service.

We considered the case of scheduling $15$ customers using a dynamic schedule. We found that the optimal interarrival time of the next customer appears to only depend on the number currently in the system provided that it is not the final customer being scheduled. If this hypothesis was proved in the general case, then it would have several potential benefits. First, it would be significantly easier to apply such a schedule in practice as the scheduler is only required to know a few optimal policies. Moreover, it would be more efficient to compute the optimal dynamic schedule as it would involve solving for a smaller number of unknown variables. However, more work is required to confirm this hypothesis.

In Chapter~\ref{chap:Comparison}, we compared the expected cost of the two schedules. As expected, the static schedule never has a smaller expected cost than the dynamic schedule. However, we found that the expected cost difference only appears to be noticeable when scheduling more than five customers. The expected percentage cost saving increases as the number of customers to be scheduled increases, but at a decreasing rate. For scheduling $15$ customers with $\gamma = 0.5$, the expected percentage cost saving is generally between $5$ and $10\%$. From our results, it does not appear that the expected percentage cost saving would ever exceed $15\%$.

In Chapter~\ref{chap:Simulation}, we applied simulation to get a fuller understanding of the differences between the two schedules. The dynamic schedule is able to adapt during service, and is thus less prone to simulation runs with extremely high cost in comparison to the static schedule. The dynamic schedule generally schedules customers to arrive earlier than the static schedule. Customers thus often have longer waiting times in the dynamic schedule than the static schedule. However, the dynamic schedule appears to be fairer as the average waiting time is approximately constant for the majority of the customers. The dynamic schedule appears to attain its lower expected cost due to its considerably lower server availability time. For $15$ customers where $\gamma = 0.5$, the median server availability time is $12 \%$ lower for the dynamic schedule.

The results presented in this thesis suggest that the dynamic schedule can often significantly outperform the static schedule. Models that ignore the potential ability to reschedule arrival times appear to be too restrictive. However, this work should be extended to more clearly understand how the rescheduling ability should be included.

A highly beneficial extension of the dynamic schedule presented in Chapter~\ref{chap:Dynamic} would include the idea of a minimum notice period. In many situations, customers are not able to arrive immediately. The dynamic schedule would be significantly more applicable if it could include a minimum value on the scheduled interarrival times. For the minimum notice period to be properly included in the schedule, the schedule must have the ability to schedule multiple customers at once rather than needing to schedule the customers one-by-one.

Other extensions on the schedules prevented in this thesis would involve adjustments to the objective function. Many papers include the server's idle time in the objective. Both the objective function for the static schedule in Chapter~\ref{chap:Static} and the expected transition cost in Chapter~\ref{chap:Dynamic} can be easily adjusted to include the cost of the server's idle time.

Another extension on the objective function could include the cost of overtime. The service could have a desired finish time, and there is a (high) cost if the service exceeds this time. This extension would be beneficial to real world services. However, applying it requires knowledge of this desired finish time parameter, which is likely heavily dependent on the specific service under consideration.