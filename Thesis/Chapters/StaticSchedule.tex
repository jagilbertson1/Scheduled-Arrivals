\chapter{Static Schedule}

\section{Aim}

The aim is to choose a schedule that minimises the expected cost of the system. The expected cost is a linear combination of the total customers' waiting time and the server's idle time. The schedule is fixed and chosen at the start of service.

\section{Assumptions}

To simplify this problem, need to make several assumptions:

\begin{itemize}[nosep]
	\item Service times are independent and identically distributed (iid)
	\item Each service time has an exponential distribution with mean service time $\mu$
	\item There is a single server
	\item The queue operates on a first in, first out (FIFO) basis
	\item Customers are punctual and arrive at their scheduled time
\end{itemize}

\section{List of Variables}

List of variables goes here

Let $w_{i}$ denote the expected waiting time of the $i$-th customer.

Let $t_{1}$ denote the scheduled arrival time of the first customer.

Let $x_{i}$ denote the time interval between the scheduled arrival of the $i$-th and the $(i + 1)$-th customer.

Let $\mu$ denote the mean service time of each customer. 

\section{Objective Function}

The expected total customers' waiting time is the sum of the individual customer's expected waiting times:
\begin{equation}
	\mathbb{E} \Big[\text{total customer's waiting time} \Big] = \sum_{i = 1}^{n} w_{i}
\end{equation}

The expected total server availability time is the sum of the $n$-th customer's scheduled arrival time, the expected waiting time of the $n$-th customer, and the expected service time of the $n$-th customer:
\begin{equation}
	\mathbb{E} \Big[\text{total server availability time} \Big] = \left( t_{1} + \sum_{i = 1}^{n - 1} x_{i} \right) + w_{n} + \mu
\end{equation}

The objective function is a linear combination of the expected total customers' waiting time and the expected total server availability time:
\begin{equation}
	\phi (\mathbf{x}) = c_{W} \sum_{i = 1}^{n} w_{i} + c_{S} \left[ t_{1} + \sum_{i = 1}^{n - 1} x_{i} + w_{n} + \mu \right]
\end{equation}

The first customer should obviously be scheduled for the start of service, so $t_{1} = 0$, which implies the objective function is:
\begin{equation}
	\phi (\mathbf{x}) = c_{W} \sum_{i = 1}^{n} w_{i} + c_{S} \left[ \sum_{i = 1}^{n - 1} x_{i} + w_{n} + \mu \right]
\end{equation}

Scale $\phi (\mathbf{x})$ by dividing by $(c_{W} + c_{S})$ and defining $\gamma = \frac{c_{S}}{c_{W} + c_{S}}$. In addition, drop the constant term $\gamma \mu$.
\begin{equation}
	\phi (\mathbf{x}) = (1 - \gamma) \sum_{i = 1}^{n} w_{i} + \gamma \sum_{i = 1}^{n - 1} x_{i} + \gamma w_{n}
	\label{eqn:StaticObjective}
\end{equation}

\subsection{Expected Customer Waiting Times}

Want to express the expected waiting time of customer $i$ as a function of the scheduled arrival time intervals $\mathbf{x}$. If there are $j$ customers in the system just prior to the arrival of customer $i$, then $w_{i} = j \mu$ by the memoryless property of the exponential distribution. Just prior to the arrival of customer $i$, the number of customers in the system must be $\in \{ 0, 1, \ldots, i - 1 \}$. Thus, the expected waiting time of customer $i$ is:
\begin{equation}
	w_{i} = \sum_{j = 0}^{i - 1} (j \mu) \cdot \mathbb{P} \Big( N (t_{i}) = j \Big) = \begin{cases} 0 & \text{where $i = 1$} \\ \sum_{j = 1}^{i - 1} (j \mu) \cdot \mathbb{P} \Big( N (t_{i}) = j \Big) & \text{where $i \geq 2$} \end{cases}
	\label{eqn:StaticWaiting}
\end{equation}

As there are no customers in the system before the start of service:
\begin{equation}
	\mathbb{P} \Big( N (t_{1}) = 0 \Big) = 1
\end{equation}

For $i \geq 2$, can express the probability of a given number of customers in the system recursively. First, consider the base case of no customers in the system just prior to the arrival of customer $i$.
\begin{align*}
	& \mathbb{P} \Big( N (t_{i}) = 0 \Big) \\
	= & \sum_{k = 1}^{i - 1} \mathbb{P} \Big( N (t_{i - 1}) = k - 1 \Big) \cdot \mathbb{P} \Big( N (t_{i}) = 0 \ \Big| \ N (t_{i - 1}) = k - 1 \Big) \\
	= & \sum_{k = 1}^{i - 1} \mathbb{P} \Big( N (t_{i - 1})= k - 1 \Big) \\
	& \ \ \ \ \ \ \cdot \mathbb{P} \Big( \text{time between $(i - 1)$-th and $i$-th arrival allows for $\geq k$ departures} \Big) \\
	= & \sum_{k = 1}^{i - 1} \mathbb{P} \Big( N (t_{i - 1}) = k - 1 \Big) \cdot \left[ \sum_{l = k}^{\infty} \frac{x_{i - 1}^{l}}{\mu^{l} \cdot l!} \exp \left( \frac{- x_{i - 1}}{\mu} \right) \right] \\
	= & \sum_{k = 1}^{i - 1} \mathbb{P} \Big( N (t_{i - 1}) = k - 1 \Big) \cdot \left[ 1 - \sum_{l = 0}^{k - 1} \frac{x_{i - 1}^{l}}{\mu^{l} \cdot l!} \exp \left( \frac{- x_{i - 1}}{\mu} \right) \right]
\end{align*}

Next, calculate the probability for $j \geq 1$:
\begin{align*}
	& \mathbb{P} \Big( N (t_{i}) = j \Big) \\
	= & \sum_{k = 0}^{i - j - 1} \mathbb{P} \Big( N (t_{i - 1}) = j + k - 1 \Big) \cdot \mathbb{P} \Big( N (t_{i}) = j \ \Big| \ N (t_{i - 1}) = j + k - 1 \Big) \\
	= & \sum_{k = 0}^{i - j - 1} \mathbb{P} \Big( N (t_{i - 1}) = j + k - 1 \Big) \\
	& \ \ \ \ \ \ \cdot \mathbb{P} \Big( \text{$k$ departures from queue between $(i - 1)$-th and $i$-th arrival} \Big) \\
	= & \sum_{k = 0}^{i - j - 1} \mathbb{P} \Big( N (t_{i - 1}) = j + k - 1 \Big) \cdot \frac{x_{i - 1}^{k}}{\mu^{k} \cdot k!} \exp \left( \frac{- x_{i - 1}}{\mu} \right)
\end{align*}

Therefore, the full recursive expression for the probability of $j$ customers in the system immediately before the arrival of customer $i$ is given by:
\begin{multline}
	\mathbb{P} \Big( N (t_{i}) = j \Big) \\
	= \begin{cases} 1 & \text{where $i = 1$, $j = 0$} \\ \sum_{k = 1}^{i - 1} \mathbb{P} \Big( N (t_{i - 1}) = k - 1 \Big) \cdot \left[ 1 - \sum_{l = 0}^{k - 1} \frac{x_{i - 1}^{l}}{\mu^{l} \cdot l!} \exp \left( \frac{- x_{i - 1}}{\mu} \right) \right] & \text{where $i \geq 2$, $j = 0$} \\ \sum_{k = 0}^{i - j - 1} \mathbb{P} \Big( N (t_{i - 1}) = j + k - 1 \Big) \cdot \frac{x_{i - 1}^{k}}{\mu^{k} \cdot k!} \exp \left( \frac{- x_{i - 1}}{\mu} \right) & \text{where $i \geq 2$, $j \geq 1$} \end{cases}
	\label{eqn:StaticProbSystem}
\end{multline}

\subsection{Computing the Objective Function}

\citet{Pegden} suggest a similar algorithm to Algorithm~\ref{alg:StaticObjective} for computing the value of the objective function given by Equation~\ref{eqn:StaticObjective} for a given vector $\mathbf{x}$ and parameters $\gamma$ and $\mu$.
\begin{algorithm}[htb]
\caption{Return $\phi (\mathbf{x})$ for a given vector $\mathbf{x}$, $\gamma$ and $\mu$}
\begin{algorithmic}
\Function{ObjectiveFunction}{$\mathbf{x}$,$\gamma$,$\mu$}
    \For {$i = 1, 2, \ldots, n$}
    	\For {$j = 0, 1, \ldots, (i - 1)$}
    		\State compute $\mathbb{P} \left( N (t_{i}) = j \right)$ by Equation~\ref{eqn:StaticProbSystem}
    	\EndFor
    \EndFor
    \For {$i = 1, 2, \ldots, n$}
    	\State compute $w_{i}$ by Equation~\ref{eqn:StaticWaiting}
    \EndFor
    \State \Return $\phi (\mathbf{x})$ computed by Equation~\ref{eqn:StaticObjective}
\EndFunction
\end{algorithmic}
\label{alg:StaticObjective}
\end{algorithm}

\section{Example Models}
\subsection{Model for Two Customers}
\label{sec:StaticTwoCust}

We consider the simplest case of this model where there are two customers to be scheduled (i.e., $n = 2$). As the first customer is scheduled to arrive at the start of service, the only unknown variable is the time interval between the first and second customers' arrivals (i.e., $x_{1}$).

By Equation~\ref{eqn:StaticWaiting}, the expected waiting times of the two customers are:
\begin{align}
	w_{1} & = 0 \\
	w_{2} & = \mu \exp \left( \frac{- x_{1}}{\mu} \right)
\end{align}

By Equation~\ref{eqn:StaticObjective}, the objective function to be minimised is:
\begin{equation}
	\phi (x_{1}) = \mu \exp \left( \frac{- x_{1}}{\mu} \right) + \gamma x_{1}
\end{equation}

This objective function is convex as:
\begin{equation}
	\forall x_{1} \ \ \phi'' (x_{1}) = \frac{1}{\mu} \exp \left( \frac{- x_{1}}{\mu} \right) \geq 0
\end{equation}

Due to the convexity of the objective function, the optimal policy that minmises $\phi (x_{1})$ can be found by solving:
\begin{equation}
	\phi' (x_{1}) = 0 \implies - \exp \left( \frac{- x_{1}}{\mu} \right) + \gamma = 0
\end{equation}

Thus, the optimal policy is:
\begin{equation}
	x_{1}^{*} = \argmin_{x_{1}} \phi (x_{1}) = - \mu \ln \gamma
\end{equation}

Therefore, the optimal arrival times of the two customers are:
\begin{equation}
	\mathbf{t}^{*} = \left( t_{1}^{*}, t_{2}^{*} \right) = \Big( 0, - \mu \ln \gamma \Big)
\end{equation}

As the server availability cost increases relative to the customer waiting cost (i.e., $\gamma$ increases), the second customer is scheduled to arrive earlier (i.e., $t_{2}^{*}$ decreases).

\subsection{Model for Three Customers}

The next simplest case is where there are three customers to be scheduled (i.e., $n = 3$). In this case, there are two unknown variables $x_{1}$ and $x_{2}$. Without loss of generalisation, can let $\mu = 1$.

By Equation~\ref{eqn:StaticWaiting}, the expected waiting times of the three customers are:
\begin{align}
	w_{1} & = 0 \\
	w_{2} & = \exp (- x_{1}) \\
	w_{3} & = \exp \Big( - (x_{1} + x_{2}) \Big) \Big[ 1 + x_{2} + \exp (x_{1}) \Big]
\end{align}

By Equation~\ref{eqn:StaticObjective}, the objective function to be minimised is:
\begin{multline}
	\phi (x_{1}, x_{2}) = \exp \Big( - (x_{1} + x_{2}) \Big) \Bigg[ 1 + x_{2} + \exp (x_{1}) \\
	+ \exp (x_{2}) \Big( 1 - \gamma + \gamma (x_{1} + x_{2}) \exp (x_{1}) \Big) \Bigg]
\end{multline}

This objective function is convex for $x_{1}, x_{2} \geq 0$ and $0 \leq \gamma \leq 1$ as:
\begin{align}
	\frac{\partial^{2} \phi}{\partial x_{1}^{2}} & = \exp \Big( - (x_{1} + x_{2}) \Big) \Big( 1 + x_{2} + \exp (x_{2}) (1 - \gamma) \Big) > 0 \\
	\frac{\partial^{2} \phi}{\partial x_{2}^{2}} & = \exp \Big( - (x_{1} + x_{2}) \Big) \Big( x_{2} + \big[ \exp (x_{1}) - 1 \big] \Big) \geq 0 \\
	\frac{\partial^{2} \phi}{\partial x_{1} \partial x_{2}} & = x_{2} \exp \Big( - (x_{1} + x_{2}) \Big) \geq 0
\end{align}

In a similar way to Section~\ref{sec:StaticTwoCust}, due to the convexity of the objective function, the optimal policy that minimises $\phi (x_{1}, x_{2})$ can be found by jointly solving:
\begin{align}
	\frac{\partial \phi}{\partial x_{1}} & = 0 \implies \exp (x_{2}) \Big( \gamma - 1 + \gamma \exp (x_{2}) \Big) = x_{2} + 1 \\
	\frac{\partial \phi}{\partial x_{2}} & = 0 \implies \exp (x_{1}) \Big( \gamma \exp (x_{2}) - 1 \Big) = x_{2}
\end{align}

Unfortunately, as \citet{Pegden} found, no algebraic solution for $(x_{1}, x_{2})$ exists to these equations, so they need to be solved numerically.

\begin{table}[htb]
\centering
\begin{tabular}{|c|c|c|c|} \hline
 $\gamma$ &  $x_{1}$ &  $x_{2}$ &  $\phi (\mathbf{x})$ \\ \hline
     1.00 &     0.00 &     0.00 &                 2.00 \\
     0.95 &     0.07 &     0.30 &                 2.04 \\
     0.90 &     0.14 &     0.41 &                 2.06 \\
     0.85 &     0.22 &     0.50 &                 2.07 \\
     0.80 &     0.30 &     0.58 &                 2.07 \\
     0.75 &     0.39 &     0.65 &                 2.05 \\
     0.70 &     0.48 &     0.73 &                 2.03 \\
     0.65 &     0.58 &     0.80 &                 2.00 \\
     0.60 &     0.68 &     0.88 &                 1.95 \\
     0.55 &     0.78 &     0.96 &                 1.89 \\
     0.50 &     0.89 &     1.05 &                 1.82 \\
     0.45 &     1.01 &     1.15 &                 1.74 \\
     0.40 &     1.13 &     1.26 &                 1.64 \\
     0.35 &     1.27 &     1.38 &                 1.53 \\
     0.30 &     1.43 &     1.51 &                 1.40 \\
     0.25 &     1.61 &     1.68 &                 1.26 \\
     0.20 &     1.83 &     1.87 &                 1.09 \\
     0.15 &     2.10 &     2.13 &                 0.90 \\
     0.10 &     2.48 &     2.49 &                 0.68 \\
     0.05 &     3.12 &     3.13 &                 0.41 \\ \hline
\end{tabular}
\caption{Optimal solution for $n = 2$ customers with mean service time $\mu = 1$ over various values of $\gamma$ found using \texttt{scipy.optimize.minimize} in Python. For $\mu \neq 1$, these solutions are the optimal values of $\mu x_{1}$ and $\mu x_{2}$.}
\end{table}













































