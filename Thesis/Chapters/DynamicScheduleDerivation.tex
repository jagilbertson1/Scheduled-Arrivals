\chapter{Dynamic Schedule Derivation}
\section{Transition Probability}
This is a derivation of the transition probability $p_{a} (k, j)$. This is the probability that there are $k - (j - 1)$ departures from the system over $a$ time units assuming the system initially has $k$ customers and no new arrivals. The customer service times are iid exponential random variables with mean $\mu$.

The probabililty is most easily derived on a case by case basis.

\subsection{Case 1 $k = 0$}
The first case is that the system initially has no customers. For any $a \geq 0$, there will be no departures from the system over $a$ time units such that
\begin{equation}
	p_{a} (k, j) = \mathbbm{1} (j = 1).
\end{equation}

\subsection{Case 2 $k \geq 1, j = 1$}
Denote the service times of the $k$ customers as $S_{1}, \ldots, S_{k}$. If $j = 1$, then $p_{a} (k, j)$ is the probability of $k$ departures from the system over $a$ time units. The sum $\sum_{i = 1}^{k} S_{i}$ has an Erlang distribution with distribution function $F (a; k)$, thus
\begin{equation}
	p_{a} (k, j) = \mathbb{P} \left( \sum_{i = 1}^{k} S_{i} \leq a \right) = F (a; k).
\end{equation}

\subsection{Case 3 $k \geq 2, 2 \leq j \leq k$}
For $2 \leq j \leq k$, $p_{a} (k, j)$ is the probability that the total service time of the first $k - (j - 1)$ customers is less that $a$, and the total service time of the first $k - (j - 1) + 1$ customers is greater than $a$,
\begin{equation}
	\begin{split}
		p_{a} (k, j)
		& = \mathbb{P} \left( \sum_{i = 1}^{k - (j - 1)} S_{i} \leq a, \sum_{i = 1}^{k - (j - 1) + 1} S_{i} > a \right) \\
		& = \mathbb{P} \left( \sum_{i = 1}^{k - (j - 1)} S_{i} \leq a, S_{k - (j - 1) + 1} > a - \sum_{i = 1}^{k - (j - 1)} S_{i} \right).
	\end{split}
\end{equation}

We condition the probability on $\sum_{i = 1}^{k - (j - 1)} S_{i} = z$, which has density function $f (z; k - (j - 1))$ and integrate over all possible values of $z$,
\begin{equation}
	\begin{split}
		p_{a} (k, j)
		& = \int_{0}^{\infty} \mathbb{P} (z \leq a, S_{k - (j - 1) + 1} > a - z) f \big( z; k - (j - 1) \big) \ d z \\
		& = \int_{0}^{a} \mathbb{P} (S_{k - (j - 1) + 1} > a - z) f \big( z; k - (j - 1) \big) \ d z \\
		& = \frac{1}{(k - j)!} \left( \frac{1}{\mu} \right)^{k - j + 1} \mathrm{e}^{\frac{- a}{\mu}} \int_{0}^{a} z^{k - j} \ d z \\
		& = \frac{1}{(k - j + 1)!} \left( \frac{a}{\mu} \right)^{k - j + 1} \mathrm{e}^{\frac{- a}{\mu}} \\
		& = F (a; k - j + 1) - F (a; k - j + 2).
	\end{split}
\end{equation}

This probability is the probability of greater than or equal to $k - (j - 1)$ departures from the system over time interval $a$ minus the probability of greater than or equal to $k - (j - 1) + 1$ departures from the system over time interval $a$.

\subsection{Case 4 $k \geq 1, j = k + 1$}
For $j = k + 1$, $p_{a} (k, j)$ is the probability that the service time of the first customer is longer than $a$ time units,
\begin{equation}
	p_{a} (k, j) = \mathbb{P} (S_{1} > a) = 1 - \mathbb{P} (S_{1} \leq a) = 1 - F (a; 1).
\end{equation}

\subsection{All Other Cases}
All other cases have zero probability,
\begin{equation}
	p_{a} (k, j) = 0.
\end{equation}

\subsection{Summary}
These results can be summarised as
\begin{equation}
	p_{a} (k, j) = \begin{cases}
		\mathbbm{1} (j = 1) & \text{for $k = 0$} \\
		F (a; k) & \text{for $k \geq 1, j = 1$} \\
		F (a; k - j + 1) - F (a; k - j + 2) & \text{for $k \geq 2, 2 \leq j \leq k$} \\
		1 - F (a; 1) & \text{for $k \geq 1, j = (k + 1)$} \\
		0 & \text{otherwise}.
	\end{cases}
\end{equation}

\section{Expected Transition Cost}
This is a derivation of the expected transition cost $R_{a} (k, j)$. This is the expected cost of transitioning from the state $(n, k)$ to the state $(n - 1, j)$ over $a$ time units if the next customer is scheduled to arrive in $a$ time units. The expected cost is a linear combination of the expected total customers' waiting times and the expected server availability time. The per unit time costs $c_{W}$ and $c_{S}$ are defined as in Chapter~\ref{chap:Static}.

In a similar way to the transition probability, the cost is most easily derived on a case by case basis.

\subsection{Case 1 $k \in \{ 0, 1 \}$}
The first case is that the system initially has either no customers or only a single customer. In this case, no customers are waiting during the transition, so the total customers waiting time is zero. The only cost is the cost of the expected server availability time during the transition. The server is available for the entire transition, so the server availability time is $a$, thus
\begin{equation}
	R_{a} (k, j) = c_{S} a.
\end{equation}

\subsection{Case 2 $k \geq 2, j = 1$}
If $j = 1$, then all $k$ customers finish service during the transition. This implies that $\displaystyle \sum_{n = 1}^{k} S_{n} \leq a$. The total customers' waiting time is a linear combination of the service times given that the summation of the service times is smaller than $a$. The server is still available for the entire transition, so the server availability time is $a$. The expected transition cost is given by
\begin{equation}
	\begin{split}
		R_{a} (k, j) & = c_{W} \sum_{i = 2}^{k} \mathbb{E} \left[ \sum_{l = 1}^{i - 1} S_{l} \ \vline \ \sum_{n = 1}^{k} S_{n} \leq a \right] + c_{S} a \\
		& = c_{W} \mathbb{E} \left[ S_{1} \ \vline \ \sum_{n = 1}^{k} S_{n} \leq a \right] \sum_{i = 2}^{k} (i - 1) + c_{S} a \\
		& = c_{W} \frac{k (k - 1)}{2} \mathbb{E} \left[ S_{1} \ \vline \ \sum_{n = 1}^{k} S_{n} \leq a \right] + c_{S} a \\
		& = c_{W} \frac{(k - 1)}{2} \mathbb{E} \left[ \sum_{n = 1}^{k} S_{n} \ \vline \ \sum_{n = 1}^{k} S_{n} \leq a \right] + c_{S} a.
	\end{split}
\end{equation}

The term $\displaystyle \mathbb{E} \left[ \sum_{n = 1}^{k} S_{n} \ \vline \ \sum_{n = 1}^{k} S_{n} \leq a \right]$ is one of the conditional expectations defined in Chapter~\ref{chap:Dynamic}. Thus,
\begin{equation}
	R_{a} (k, j) = c_{W} \frac{G (a; k) (k - 1)}{2} + c_{S} a.
\end{equation}

\subsection{Case 3 $k \geq 2, 2 \leq j \leq k$}
For $2 \leq j \leq k$, the first $k - (j - 1)$ customers finish service during the transition, but customer $k - (j - 1) + 1$ does not. This implies that $\displaystyle \sum_{n = 1}^{k - (j - 1)} S_{n} \leq a$ and $\displaystyle \sum_{n = 1}^{k - (j - 1) + 1} S_{n} > a$. The last $j - 2$ customers wait for the entire transition. In addition, the server is available for the entire transition. The expected transition cost is given by
\begin{align*}
	R_{a} (k, j)
	= & \ c_{W} \sum_{i = 2}^{k - (j - 2)} \mathbb{E} \left[ \sum_{l = 1}^{i - 1} S_{l} \ \vline \ \sum_{n = 1}^{k - (j - 1)} S_{n} \leq a, \sum_{n = 1}^{k - (j - 1) + 1} S_{n} > a \right] \\
	& + c_{W} \sum_{i = 1}^{j - 2} a + c_{S} a \\
	= & \ c_{W} \mathbb{E} \left[ S_{1} \ \vline \ \sum_{n = 1}^{k - (j - 1)} S_{n} \leq a, \sum_{n = 1}^{k - (j - 1) + 1} S_{n} > a \right] \sum_{i = 2}^{k - (j - 2)} (i - 1) \\
	& + c_{W} a (j - 2) + c_{S} a
\end{align*}

\begin{equation}
	\begin{split}
		R_{a} (k, j)
		= & \ c_{W} \frac{ \big[ k - (j - 1) \big] \big[ k - (j - 2) \big]}{2} \mathbb{E} \left[ S_{1} \ \vline \ \sum_{n = 1}^{k - (j - 1)} S_{n} \leq a, \sum_{n = 1}^{k - (j - 1) + 1} S_{n} > a \right] \\
		& + c_{W} a (j - 2) + c_{S} a \\
		= & \ c_{W} \frac{ \big[ k - (j - 2) \big]}{2} \mathbb{E} \left[ \sum_{n = 1}^{k - (j - 1)} S_{n} \ \vline \ \sum_{n = 1}^{k - (j - 1)} S_{n} \leq a, \sum_{n = 1}^{k - (j - 1) + 1} S_{n} > a \right] \\
		& + c_{W} a (j - 2) + c_{S} a.
	\end{split}
\end{equation}

The term $\displaystyle \mathbb{E} \left[ \sum_{n = 1}^{k - (j - 1)} S_{n} \ \vline \ \sum_{n = 1}^{k - (j - 1)} S_{n} \leq a, \sum_{n = 1}^{k - (j - 1) + 1} S_{n} > a \right]$ is another of the conditional expectations defined in Chapter~\ref{chap:Dynamic}. Thus,
\begin{equation}
	\begin{split}
		R_{a} (k, j)
		= & \ c_{W} \left[ \frac{H \big( a; k - (j - 1) \big) \big[ k - (j - 2) \big]}{2} + a (j - 2) \right] + c_{S} a \\
		= & \ c_{W} \left[ \frac{a \big[ k - (j - 1) \big]}{2} + a (j - 2) \right] + c_{S} a \\
		= & \ c_{W} \frac{a (k + j - 3)}{2} + c_{S} a.
	\end{split}
\end{equation}

\subsection{Case 4 $k \geq 2, j = (k + 1)$}
For $j = k + 1$, no customers finish service during the transition. All customers except the first customer wait for the entire transition. The server is available for the entire transition. The expected transition cost is given by
\begin{equation}
	R_{a} (k, j) = c_{W} \sum_{i = 1}^{k - 1} a + c_{S} a = c_{W} a (k - 1) + c_{S} a.
\end{equation}

\subsection{Summary}
In order to compare the dynamic schedule with the static schedule, need to scale these costs by dividing by $(c_{S} + c_{W})$ and defining $\gamma = \frac{c_{S}}{c_{S} + c_{W}}$. These results can be summarised as
\begin{equation}
	R_{a} (k, j) = \begin{cases}
		\gamma a & \text{for $k \in \{ 0, 1 \}$} \\
		(1 - \gamma) \frac{G (a; k) (k - 1)}{2} + \gamma a & \text{for $k \geq 2, j = 1$} \\
		(1 - \gamma) \frac{a (k + j - 3)}{2} + \gamma a & \text{for $k \geq 2, 2 \leq j \leq k + 1$}.
	\end{cases}
\end{equation}

















































