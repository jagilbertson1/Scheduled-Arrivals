\chapter{Literature Review}

Health care providers are under a great deal of pressure to improve service quality and efficiency \citep{Goldsmith}. There is a large body of literature studying the potential of appointment systems to reduce patient waiting times, and waiting room congestion. \citet{Fomundam}, and \citet{Cayirli} provide comprehensive surveys of research on appointment scheduling. There is a fundamental trade-off in appointment policies. If patients are scheduled to arrive close together, they experience long waiting times. However, if appointment times are spread further apart, the doctor's idle time increases.

Most of the papers on scheduled arrivals in health care can be classed into two categories. Those that design algorithms to find good schedules, and those that evaluate schedules using simulation. While simulation studies can easily model complicated patient flows, queuing models often provide more generic results than simulation \citep{Green}.

The foundation paper on modeling queues with scheduled arrivals is \citet{Bailey}. Bailey proposes that customers' waiting times can be reduced without a significant increase in doctor's idle time. The Bailey rule, which is commonly referenced in literature, is that patients should be scheduled to arrive at fixed intervals with two patients scheduled to arrive at the start of service. Bailey found that a great deal of time wasted by patients could be reduced without a signficant increase in the doctor's idle time. Under the Bailey rule, patients with late appointments will wait longer than those with early appointments. This lack of uniformity might be undesirable due to issues of fairness. 

\citet{Pegden} extend on Bailey's paper. They present an algorithm to iteratively determine the optimal arrival times for $n$ patients that need to be scheduled. The optimal arrival times are those that minimise a weighted sum of the expected patients' waiting time and the expected doctor's idle time. Pegden and Rosenshine prove that their objective function is convex for $n \leq 4$, thus their algorithm finds the optimal schedule. While they conjecture that the objective function is convex for $n \geq 5$, it hasn't been proven.

\citet{Stein} apply Pegden and Roshenshine's model to obtain numerical results for situations with more than three patients. The optimal times between successive patients become near constant as $n$ grows. This is the often observed dome-shape. Optimal appointment intervals exhibit a common pattern where they initially increase towards the middle of a session, and then decrease. Stein and C\^{o}te simplify the model by requiring the intervals between arriving patients to be held constant. This realistic restriction (used commonly in the literature) makes the model more easily applicable in practice without significant altering the results.

Stein and C\^{o}te apply queuing theory results to solve the model for the optimal arrival interval assuming the queue reaches its steady state distribution. This assumption greatly reduces the computation required. However, in practice, it is common to find services that never reach steady state. \citet{Babes} attempted to apply steady state queuing theory, but found their results tended to be very different from those observed in real operation.

These key papers by Bailey, Pegden and Rosenshine, and Stein and C\^{o}te provide the basis for a more realistic exploration of health care systems. \citet{Delaurentis} point out that patient no-shows can lead to a waste of resources. \citet{Mendel} incorporates the probability of a patient not showing up into the model presented by Pegden and Rosenshine. Unsurprisingly, no-shows lead to lower expected waiting times for patients who do show up.

The presence of walk-ins (regular and emergency) can disrupt a schedule. \citet{Gupta} propose a system where non-routine requests are superimposed on top of routine scheduled requests. \citet{Fiems} investigate the effect of emergency requests on the waiting times of scheduled patients. Fiem models a system with deterministic service times and discrete time. Despite this research, Cayirli and Veral suggest that walk-ins are neglected in most studies. Further research could investigate their effect on optimal arrival times.

\citet{Mondschein} observe that the majority of the literature assumes that demand is exogenous and independent of patients' waiting times. These papers assume the total number of patients $n$ is fixed and independent of waiting times. The vast majority of servers are now private (including medical servers), so face competitive environments. Mondschein and Weintraub thus present a model where demand depends on the patients' expected waiting time.

Simulation is a useful tool to analyse the effectiveness of appointment policies. \citet{Kao} use simulation to compliment their results obtained from queuing theory. \citet{Ho} study the performance of eight different appointment rules under different scenarios. They find that no rule will perform well under all circumstances.

Case studies can test the real world applications of an appointment system. While they lack generalisation, they are necessary to compliment the theoretical research. \citet{Rockart} show individual block systems lead to more punctual doctors and patients, and less no-shows. \citet{Walter} indicates that the simple grouping of inpatients and outpatients results in a substantial improvement in doctor's idle time.

Unfortunately, \citet{Cayirli} lament that despite much published work, the impact of appointment systems on outpatient clinics has been limited. Doctors are often unwilling to change their old habits. \citet{O'Keefe} had their proposed appointment system of classifying patients rejected by staff. \citet{Huarng} were unable to implement their system due to staff resistance. \citet{Bennett} found their recommendations weren't implemented successfully. Future research must attempt to develop models that will be accepted and implemented in real health care services.
